\documentclass[a4paper,11pt]{article}

\usepackage[margin=2cm]{geometry}
\usepackage{fancyhdr}
\usepackage{hyperref}
\usepackage{color}
\usepackage{graphicx} 
\usepackage{amsmath}
\usepackage{listings}

\definecolor{dkgreen}{rgb}{0,0.6,0}
\definecolor{gray}{rgb}{0.5,0.5,0.5}
\definecolor{lightgray}{rgb}{0.95,0.95,0.95}
\definecolor{mauve}{rgb}{0.58,0,0.82}

\lstset{frame=tb,
  language=python,
  backgroundcolor=\color{lightgray}, 
  aboveskip=3mm,
  belowskip=3mm,
  showstringspaces=false,
  columns=flexible,
  basicstyle={\small\ttfamily},
  numbers=none,
  numberstyle=\tiny\color{gray},
  keywordstyle=\color{blue},
  commentstyle=\color{dkgreen},
  stringstyle=\color{mauve},
  breaklines=true,
  breakatwhitespace=true,
  tabsize=3,
  frame=single
}

\definecolor{DarkBlue}{rgb}{0.00,0.00,0.50}
\hypersetup{
    linkcolor   = DarkBlue,
    anchorcolor = DarkBlue,
    citecolor   = DarkBlue,
    filecolor   = DarkBlue,
    pagecolor   = DarkBlue,
    urlcolor    = DarkBlue,
    colorlinks  = true
}

\usepackage[round]{natbib}

\addtocounter{secnumdepth}{1}

\bibliographystyle{plainnat}


\usepackage{palatino}

% Commands to allow insertion of named comments:
\renewcommand{\DH}[1]{{\color{blue} DH: #1}}
\newcommand{\RF}[1]{{\color{green} RF: #1}}
\newcommand{\GR}[1]{{\color{red} GR: #1}}
\newcommand{\SP}[1]{{\color{magenta} SP: #1}}
\newcommand{\EM}[1]{{\color{orange} EM: #1}}
\newcommand{\MGCM}[1]{{\color{cyan} MGCM: #1}}

% Commands to make cross-links to the glossary
\newcommand{\API}{\hyperref[glos:API]{API}}

% Allow lots of floats on one page
\renewcommand{\textfraction}{0.05}
\renewcommand{\topfraction}{0.8}
\renewcommand{\bottomfraction}{0.8}

% A little maths
\newcommand{\DoF}{\ensuremath{\mathrm{DoF}}}

\title{Met Office/NERC Cloud Model (MONC): User documentation.}

\author{Nick Brown, Ben Shipway, Adrian Hill}

\date{April 2015} % Assuming that's when we're aiming for!

%% Note about the context of this document:
%% This document should be a companion to the software and
%% scientific papers which will be published in journals.  It should
%% get into the meat of how to run MONC.
%% I think it's a good idea where appropriate to refer to the 
%% similarities with the old LEM.  This wil help existing users 
%% get up to speed more quickly.  However, we want this to be a 
%% stand alone document and so should avoid referencing and
%% depending on the LEM documentation 


\begin{document}

\maketitle
\newpage
\tableofcontents
\newpage

\section{Configuration files}




\section{Setting up initial profiles}

\begin{table}
\protect\caption{Configuration of piecewise linear initial profiles}
\begin{tabular}{|c|c|c|}
\hline 
Configuration variable name & configuration type & description\tabularnewline
\hline 
\hline 
l\_init\_pl\_X & logical & if .true. then will set up piecewise  \tabularnewline
 &  & linear initial profiles for X=$\theta$,u,v or q\tabularnewline
\hline 
l\_thref & logical & if .true. then will set up piecewise linear \tabularnewline
\hline 
 &  & profile for thref\tabularnewline
\hline 
l\_matchthref & logical & if .true. then profile for thref will be the same\tabularnewline
 &  &  as that set for the initial $\theta$ profile\tabularnewline
\hline 
names\_init\_pl\_q & string array & list of q variables that should be initialized\tabularnewline
\hline 
z\_init\_pl\_X & float array & list of heights at the nodes of the profile\tabularnewline
 &  & of X=$\theta$,u or v\tabularnewline
\hline 
z\_init\_pl\_q{[\textless q-name\textgreater]} & float array & list of heights at the nodes of the q profile\tabularnewline
 &  & with name \textless q-name\textgreater\tabularnewline
\hline 
f\_init\_pl\_X & float array & list of values at the nodes of the profile\tabularnewline
 &  & of X=$\theta$,u or v\tabularnewline
\hline 
f\_init\_pl\_q{[\textless q-name\textgreater]} & float array & list of values at the nodes of the q profile\tabularnewline
 &  & with name \textless q-name\textgreater\tabularnewline
\hline 
\end{tabular}
\end{table}

\begin{lstlisting}[caption={Example configuration snippet for initial profiles from BOMEX\_config}]
# Initial profiles for $\theta$ (K) and water vapour (kg/kg)
# for the BOMEX shallow cumulus test case
l_init_pl_theta = .true.
z_init_pl_theta = 0.0, 520.0, 1480., 2000., 3000.
f_init_pl_theta = 298.7, 298.7, 302.4, 308.2, 311.85
l_init_pl_q     = .true.
names_init_pl_q = vapour
z_init_pl_q[vapour] = 0.0, 520.0, 1480., 2000., 3000.
f_init_pl_q[vapour] = 17.0e-3, 16.3e-3, 10.7e-3, 4.2e-3, 3.0e-3
\end{lstlisting}

\section{Surface boundary conditions}

Surface boundary conditions for $\theta$ and $q_v$ variables are determined by
the options outlined in table \ref{table:surface_conditions}.  These can use
either prescribed surface values (input as temperature and relative humidity),
or else prescribed surface fluxes (input as sensible and latent heat).
Conditions can vary in time and either be read in through a netcdf file, or
else directly as a list of inputs through the configuration file.  If time
independent forcing is required, then the first value in the list is used
throughout the simulation.

\begin{table}\label{table:surface_conditions}
\protect\caption{Configuration of surface boundary conditions.}
\begin{tabular}{|c|c|c|}
\hline 
Configuration variable name & configuration type & description\tabularnewline
\hline 
\hline 
use\_surface\_boundary\_conditions & logical & if .true. then will enable \tabularnewline
 &  & surface boundary conditions for $\theta$ and $q_v$\tabularnewline
\hline 
type\_of\_surface\_boundary\_conditions & integer & if = 0, use prescribed surface fluxes \tabularnewline
 &  & if = 1, use prescribed surface values \tabularnewline
 &  & for $\theta$ and $q_v$\tabularnewline
\hline
use\_time\_varying\_surface\_conditions & logical & if .true. then will enable \tabularnewline
 &  & time varying boundary conditions\tabularnewline
 &  & if .false. then the first entry in the\tabularnewline
 &  & following variables will be used\tabularnewline
 &  &  throughout the simulation. \tabularnewline
\hline  
surface\_conditions\_file & string & netcdf filename\footnote{To Do: Need to document
the format for these files} for input
values. \tabularnewline
 &  & If string is not provided or is 'None',  \tabularnewline 
 &  & then values will be used \tabularnewline 
 &  & from the configuration variables below. \tabularnewline 
\hline  
surface\_boundary\_input\_times & real array & an array of times at which \tabularnewline
 &  & values are specified \tabularnewline
 &  &  [seconds from start of simulation]\tabularnewline 
\hline  
surface\_temperatures & real array & an array of values to use for \tabularnewline
 &  & surface temperatures [K]\tabularnewline
\hline 
surface\_humiditys & real array & an array of values to use for surface\tabularnewline
 &  & relative humidity [\%]\tabularnewline 
 &  & (if temperature values are provided,\tabularnewline
 &  &  but not humidity values,  then 100\% humidity \tabularnewline
 &  & is assumed, i.e. Sea surface)\tabularnewline
\hline
surface\_sensible\_heat\_flux & real array & an array of values to use for \tabularnewline
 &  & surface flux of sensible heat [Wm$^{-2}$]\tabularnewline
\hline 
surface\_latent\_heat\_flux & real array & an array of values to use for \tabularnewline
 &  & surface flux of latent heat [Wm$^{-2}$]\tabularnewline
\hline 
\end{tabular}
\end{table}


\section{Time-independent forcing}

Time-independent piecwise linear forcings profiles can be input in a similar
manner to initial profiles.  



\begin{table}
\protect\caption{Configuration of time independent forcing profiles}
\begin{tabular}{|c|c|c|}
\hline 
Configuration variable name & configuration type & description\tabularnewline
\hline 
\hline 
l\_constant\_forcing\_X & logical & if .true. then will set up piecewise linear\tabularnewline
 &  & forcing profiles for X=$\theta$,u,v or q\tabularnewline
\hline 
names\_constant\_forcing\_q & string array & list of q variables that should be forced\tabularnewline
\hline 
z\_force\_pl\_X & float array & list of heights for the nodes of the forcing\tabularnewline
 &  & profile of X=$\theta$,u or v\tabularnewline
\hline 
z\_force\_pl\_q[\textless q-name\textgreater ] & float array & list of heights for the nodes of the q forcing\tabularnewline
 &  & profile with name \textless q-name\textgreater \tabularnewline
\hline 
f\_force\_pl\_X & float array & list of values for the nodes of the forcing\tabularnewline
 &  & profile of X=$\theta$,u or v\tabularnewline
\hline 
f\_force\_pl\_q[\textless q-name\textgreater ] & float array & list of values for the nodes of the q forcing\tabularnewline
 &  & profile with name \textless q-name\textgreater \tabularnewline
\hline 
units\_theta\_force & string & units for the forcing data for $\theta$\tabularnewline
\hline 
units\_q\_force & string array & units for the forcing data for the q variables\tabularnewline
\hline 
constant\_forcing\_type\_X & integer & Determines how the forcing profiles should \tabularnewline
 &  & be used (see text):\tabularnewline
 &  & 0 = TENDENCY\tabularnewline
 &  & 1 = RELAXATION\tabularnewline
 &  & 2=INCREMENTS\tabularnewline
\hline 
convert\_input\_theta\_from\_temperature & logical & if .true. then convert then input data is\tabularnewline
 &  & converted from temperature to potential\tabularnewline
 &  & temperature\tabularnewline
\hline 
convert\_input\_specific\_to\_mixingratio & logical & if .true. then convert input data from\tabularnewline
 &  & a specific humidity to mixing ratio\tabularnewline
\hline 
\end{tabular}
\end{table}

\begin{minipage}{\linewidth}
\begin{lstlisting}[caption={Example configuration snippet for time independent
    forcing from BOMEX\_config}]
# Time dependent forcing theta (K) and water vapour (kg/kg)
# for the BOMEX shallow cumulus test case
l_constant_forcing_theta = .true.
l_constant_forcing_q     = .true.
l_constant_forcing_u     = .false.
l_constant_forcing_v     = .false.
units_theta_force        = K/day
z_force_pl_theta         = 0.0, 1500.0, 2500.0, 3000.
f_force_pl_theta         = -2.0, -2.0, 0.0, 0.0
names_constant_forcing_q = vapour
units_q_force            = kg/kg/s
z_force_pl_theta         = 0.0, 300.0, 500.0, 3000.
f_force_pl_theta         = -1.2e-8, -1.2e-8, 0.0, 0.0
constant_forcing_type_theta = 0
constant_forcing_type_q     = 0
convert_input_theta_from_temperature  = .true.
convert_input_specific_to_mixingratio = .true.
\end{lstlisting}
\end{minipage}

\section{Radiation}

\section{Subsidence}

\section{Random perturbations}




\end{document}
